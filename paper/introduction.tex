
Modern multi-band photometric sky surveys aim to deliver measurements accurate\footnote{Except for zeropoint offsets from the AB magnitude scale, discussed in Section XXX.} at the 1\% (0.01 mag) level, to 
enable cosmological and other high-precision measurements \cite[e.g., the Vera Rubin Observatory Legacy Survey of Space and Time,][]{LSSToverview}. 
Photometric data are usually calibrated using sets of standard stars whose brightness is known from previous work.
One of the largest catalogs with sub-percent measurement precision and optical multi-band $ugriz$ photometry
was constructed by averaging multi-epoch data for about a million stars collected by the Sloan Digital Sky Survey \citep[SDSS,][]{York2000}  in a 300 deg$^2$ region known as SDSS Stripe 82 \citep[][hereafter \pO]{Ivez07}. 
The SDSS $ugriz$ photometric system is now in use at many observatories worldwide and this catalog\footnote{Available from http://faculty.washington.edu/ivezic/sdss/catalogs/stripe82.html}
(hereafter \pOc) has been used both for calibration and testing of other surveys. 

After the completion of \pOc, SDSS has obtained additional imaging data, about 2-3 times more measurements 
per star depending on its sky position within Stripe 82. This increased number of data points can result in averaged photometry with
1.4-1.7 times smaller random errors (precision) than in the original catalog (and about three times as small as for individual 
SDSS runs). In addition, the availability of photometric data from recent wide-field surveys such as the 
Dark Energy Survey \citep{2016MNRAS.460.1270D}, Pan-STARRS \citep{2010SPIE.7733E..0EK} and Gaia \citep{2018A&A...616A...1G}, enables a much more  
detailed and robust cross-calibration, including
correcting for residual photometric zeropoint eerrors in SDSS flat-fielding (e.g., a saw-tooth pattern, as a function of Declination,
was reported by \citealt{2013A&A...552A.124B}; see their Fig.~23). These are the main reasons that motivated us to construct an updated version of the SDSS Standard Star Catalog.
 

We describe datasets used in our analysis in \S2, and the construction of the new catalog and its analysis in \S3. Our results are summarized and discussed in \S4. 